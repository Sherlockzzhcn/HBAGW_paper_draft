\documentclass[a4paper,11pt]{article}
% \pdfoutput=1 % if your are submitting a pdflatex (i.e. if you have
             % images in pdf, png or jpg format)
\usepackage{jcappub} % for details on the use of the package, please
                     % see the JCAP-author-manual
\usepackage[T1]{fontenc} % if needed
\usepackage{float} 
\usepackage{lmodern}
\usepackage{booktabs}
\usepackage{siunitx}
\usepackage[english]{babel}
\addto\captionsenglish{
  \renewcommand{\figurename}{Figure}
  \renewcommand{\tablename}{Table}
}
\usepackage[utf8]{inputenc}
\usepackage{natbib}
\usepackage[colorlinks=true, citecolor=blue, urlcolor=blue, linkcolor=blue]{hyperref} 
\usepackage{graphicx}
\usepackage{subfigure}% Include figure files
\usepackage[justification=raggedright]{caption}
\usepackage{tabularx}
\usepackage{dcolumn}% Align table columns on decimal point
\usepackage{bm}
% \usepackage{ulem}

\newcommand{\ms}{M_\odot}
\newcommand{\bmt}{{\bm{\theta}}}
\newcommand{\bmT}{{\bm{\Theta}}}
\newcommand{\bmH}{{\bm{H}}}
\newcommand{\rmd}{{\rm{d}}}
\newcommand{\ZW}[1]{\textcolor{magenta}{$\mathcal{ZW}$:~#1}}

 \newcommand{\LS}[1]{\textcolor{magenta}{\bf #1}} 
% \newcommand{\ZW}[2]{{\color{blue} \sout{#1} ZW: {#2}}} % For comment
% \newcommand{\Reply}[1]{{\bf\color{blue} #1}}

\title{Inferring neutron-star Love-Q relations from gravitational waves in the
hierarchical Bayesian framework}

%% %simple case: 2 authors, same institution
%% \author{A. Uthor}
%% \author{and A. Nother Author}
%% \affiliation{Institution,\\Address, Country}

% more complex case: 4 authors, 3 institutions, 2 footnotes
\author[a]{Zhihao Zheng,}
\author[b,c,1]{Ziming Wang\note{Corresponding authors.},}
\author[d]{Jinwen Deng,}
\author[b,c]{Yiming Dong,}
\author[c,e,1]{and Lijing Shao}


% The "\note" macro will give a warning: "Ignoring empty anchor\cdots"
% you can safely ignore it.

\affiliation[a]{School of Yuanpei, Peking University, Beijing 100871, China}
\affiliation[b]{Department of Astronomy, School of Physics, Peking University,
Beijing 100871, China}
\affiliation[c]{Kavli Institute for Astronomy and Astrophysics, Peking
University, Beijing 100871, China}
\affiliation[d]{School of Physics, Peking University, Beijing 100871, China}
\affiliation[e]{National Astronomical Observatories, Chinese Academy of
Sciences, Beijing 100012, China}

% \affiliation[a]{One University,\\some-street, Country}
% \affiliation[b]{Another University,\\different-address, Country}
% \affiliation[c]{A School for Advanced Studies,\\some-location, Country}

% e-mail addresses: one for each author, in the same order as the authors
\emailAdd{zhzheng@stu.pku.edu.cn}
\emailAdd{zwang@pku.edu.cn}
\emailAdd{deng\_le0@stu.pku.edu.cn}
\emailAdd{ydong@pku.edu.cn}
\emailAdd{lshao@pku.edu.cn}

\abstract{Despite the large uncertainties in the equation of state for neutron
stars (NSs), a tight universal ``Love-Q'' relation exists between their
dimensionless tidal deformability, $\Lambda$, and the dimensionless quadrupole
moment, $Q$.  However, this relation has not yet been directly measured through
observations.  Gravitational waves (GWs) emitted from binary NS (BNS)
coalescences provide an avenue for such a measurement.  In this study, we adopt
a hierarchical Bayesian framework and combine multiple simulated GW events to
measure the Love-Q relation.  We simulate 1000 GW sources and select 20 events
with the highest signal-to-noise ratios and NS spins for the analysis.  By
inspecting four parameterization models of the Love-Q relation, we observe
strong correlations between the model parameters.  We verify that a linear
relation between $\ln\Lambda$ and $\ln Q$ is practically sufficient to describe
the Love-Q relation with the  precision expected from next-generation GW
detectors.  Furthermore, we utilize the inferred Love-Q relation to test
modified gravity. Taking the dynamical Chern-Simons gravity as an example, our
results suggest that the characteristic length can be constrained to $10\,
\mathrm{km}$ or less with future GW observations. 
}

\begin{document}
\maketitle
\flushbottom

%=============================
\section{Introduction}
\label{sec:introducion}
%=============================

Thanks to their extreme densities and strong gravitational fields, neutron stars
(NSs) serve as natural laboratories for studying nuclear and gravitational
physics (see e.g., Ref.~\cite{Shao:2022koz}). Inferred quantities from
electromagnetic observations, such as the observed maximum 
mass~\cite{Ozel:2010bz, Hebeler:2013nza, Antoniadis:2013pzd} and mass-radius 
relation~\cite{Lattimer:2006xb, Steiner:2010fz, Ozel:2010fw, Ozel_2013,
Guver:2013xa}, allow one to probe the properties of nuclear matter at densities
exceeding the nuclear saturation density. Additionally, the observation of
GW170817 has opened up a new window for investigating NS properties using
gravitational waves (GWs) emitted from binary NS (BNS)
coalescences~\cite{LIGOScientific:2017vwq, LIGOScientific:2018cki, 
LIGOScientific:2018hze}. Some of the NS properties, including the tidal
deformability and the spin-induced quadrupole moment, contribute to the GW
emission ~\cite{Poisson:1997ha, Vines:2011ud, Favata:2013rwa, Wade:2014vqa,
Samajdar:2019ulq, Abac:2023ujg} and therefore could be measured from GW
observations~\cite{Harry:2018hke, Baiotti:2019sew, Chatziioannou:2020pqz,
Agathos:2015uaa, Krishnendu:2017shb, Krishnendu:2019tjp, Gao:2021uus}. 

In a BNS system, each NS is deformed due to the gravitational field of its
companion, leading to an induced mass quadrupole moment~\cite{Hinderer:2007mb,
Damour:2009vw}. The effect is characterized by the tidal deformability
$\Lambda=2k_2/(3C^5)$, where $k_2$ is the tidal Love number and $C$ is the NS
compactness~\cite{Flanagan:2007ix}. Also, a rotating NS experiences another
deformation due to its spin, which induces the so-called  spin-induced
quadrupole moment, $\mathcal{Q}=-Q\chi^2 m^3$, where $m$ is the mass of the NS,
$\chi$ and $Q$ are the dimensionless spin and quadrupole moment
respectively~\cite{Hartle:1968, Laarakkers:1997hb}.  These properties can
provide an insight into the internal structure of NSs and bring opportunities to
test strong-field gravity~\cite{Akmal:1998cf, Demorest:2010bx, Ozel:2016oaf,
NANOGrav:2019jur, Li:2020wbw, Hu:2021tyw, Dong:2023vxv}. 

The internal structure of NSs depends on both the underlying gravity theory and
the equation of state (EOS); the latter describes the relation between pressure
and density of the NS matter. Despite starting to place constraints on the EOS,
current observations are in general not accurate enough to distinguish between
various EOS candidates~\cite{Lattimer:2006xb, Steiner:2010fz, Ozel:2010fw,
Hebeler:2013nza, Ozel_2013}.  This means that when it comes to testing
strong-field gravity, one will encounter a degeneracy between the EOS and the
underlying gravity theory~\cite{Yagi:2013bca, Yagi:2013awa, Shao:2017gwu,
Shao:2019gjj, Silva:2020acr, Shao:2022koz}.  One way to break this degeneracy is
to find universal relations among NS properties.  Yagi and
Yunes~\cite{Yagi:2013bca, Yagi:2013awa} found such a universal relation between 
$I$ (moment of inertia), $\Lambda$ (tidal Love number) and $Q$ (quadrupole
moment), assuming the validity of general relativity (GR).  On the one hand,
this ``I-Love-Q'' relation is insensitive to EOS uncertainties with variations
of about $1\%$ or less for different EOSs.  On the other hand, the relation can
deviate from the GR prediction in modified gravity theories~\cite{Yagi_2017,
Gupta:2017vsl, Yunes:2025xwp}, enabling an EOS-insensitive test of gravity.
Among the I-Love-Q trio, the Love-Q relation can be probed with GW observations,
since both $\Lambda$ and $Q$ affect the GWs emitted by BNS systems.  If the
Love-Q relation in GR is adopted as a prior in the waveform model, the number of
independent parameters can be reduced, which helps to better estimate the spin
parameters of BNSs~\cite{Yagi:2013bca, LIGOScientific:2018cki,
LIGOScientific:2018hze, LIGOScientific:2020aai}. The existence of such universal
relations, which now include many others~\cite{Lau:2009bu, Yagi:2013sva,
Maselli:2013mva, Pani:2015nua, Yagi:2016qmr, Gao:2023mwu, Hu:2025gab}, may also
imply a ``no-hair-theorem-like'' behavior for NSs, bringing us a new insight
into fundamental physics. 

Future next-generation (XG) ground-based GW detectors, including the Cosmic
Explorer (CE)~\cite{Reitze:2019iox, Reitze:2019dyk} and the Einstein Telescope
(ET)~\cite{Punturo:2010zz, Hild:2010id, Sathyaprakash:2012jk, ET:2025xjr}, are
expected to detect many more GW signals, up to about $10^5$--$10^6$ events per
year ~\cite{LIGOScientific:2017zlf, Sathyaprakash:2019yqt, Kalogera:2021bya,
Samajdar:2021egv}, thanks to their increased sensitivity and lower cutoff
frequencies. These high-precision GW observations allow us to treat $\Lambda$
and $Q$ as independent parameters in the waveform model, to be measured directly
from GWs. This enables further constraints on the Love-Q relation from an
observational perspective.  Samajdar and Dietrich~\cite{Samajdar:2020xrd} have
first performed an analysis discussing the prospects of constraining Love-Q 
relation with GW observations, where a weighted linear regression was performed.
While the results are very valuable, as we will show below, such a treatment
might miss possible degeneracy and non-Gaussianity in the posteriors of
$\Lambda$ and $Q$. 

For the first time, our study adopts the hierarchical Bayesian framework to
infer the Love-Q relation with XG GW observations.  This framework has been
successfully applied in population studies of compact binary coalescences, and
in probing the EOS of NSs~\cite{Mandel:2009nx, Mandel:2009pc, Adams:2012qw,
Lackey:2014fwa, Mandel:2018mve, Golomb:2021tll, KAGRA:2021duu, Wang:2024xon}. 
Regarding the fitting parameters in the Love-Q relation as hyperparameters, the
hierarchical Bayesian framework separates the inferences of these
hyperparameters and the single-event parameters into two layers to avoid a
direct, high-dimensional inference for all unknown parameters. It significantly
reduces the computational cost in combining information from multiple events.
Also, the construction of the quasi-likelihood function in this framework
incorporates the full shape of the posterior in single-event inference beyond
Gaussianity, thus utilizes the information contained therein in a more
comprehensive way.  

In our implementation, we simulate 1000 GW events based on population models of
NSs~\cite{Fishbach:2018edt, Farrow:2019xnc, Samajdar:2020xrd} and select the 20
loudest events for analysis.  We find that the primary information for
constraining the Love-Q relation comes from the 10 loudest GW events, consistent
with results found in previous studies~\cite{Lackey:2014fwa}. In the pioneering
work of Samajdar and Dietrich~\cite{Samajdar:2020xrd}, a linear relation between
$\ln\Lambda$ and $\ln Q$ is adopted. By further considering four polynomial
models from linear to quartic terms in fitting the relation, we quantitatively
show that the linear relation is accurate enough when constraining the Love-Q
relation with GWs.  Additionally, we apply the inferred Love-Q relation in
gravity tests.  Taking the dynamical Chern-Simons (dCS) gravity as an example,
we find that the characteristic length $\xi_{\rm CS}^{1/4}$---with $\xi_{\rm
CS}$ the theory parameter---can be limited to $\lesssim 10\,{\rm km}$ with
future GW observations. 

This paper is organized as follows. In section~\ref{sec:framework} we construct 
the hierarchical Bayesian framework and derive the posterior of the
hyperparameters.  The simulation procedure is explained in
section~\ref{sec:simulation}.  We present the results of our inference and
discuss the differences between different Love-Q parameterization models in
section~\ref{sec:results}.  We compare our inference results with the
predictions in the dCS gravity in section~\ref{sec:dCS}. Finally, we conclude in
section~\ref{sec:conclusion}.


%=============================
\section{Hierarchical Bayesian Inference of Love-Q Relation}
\label{sec:framework}
%=============================

%=============================
\subsection{Polynomial Models of Love-Q Relation} 
\label{subsec:framework_parameterization}
%=============================

Yagi and Yunes~\cite{Yagi:2013bca, Yagi:2013awa, Yagi_2017} fit the Love-Q
relation with a quadratic polynomial model as,
%--
\begin{equation}
\label{5-d_Love_Q_eq}
    \ln Q_{5}=a_5 + b_5 \ln \Lambda + c_5 \ln^2\Lambda + d_5 \ln^3\Lambda + e_5
    \ln^4 \Lambda\,,
\end{equation}
%--
where the dimensionless fitting coefficients are $a_5=0.1940$, $b_5=0.09163$,
$c_5=0.04812$, $d_5=-4.283\times 10^{-3}$ and $e_5=1.245\times
10^{-4}$~\cite{Yagi_2017}, and the lower subscript ``$5$'' indicates five model
parameters.  They found that such a relation applies to most of the EOSs with 
relative deviation less than $1\%$ in GR. When constraining the Love-Q relation
with GWs, \citet{Samajdar:2020xrd} adopted a linear model,
%--
\begin{equation}
\label{2-d_Love_Q_eq}
    \ln Q_{2} = a_2 + b_2 \ln \Lambda\,,
\end{equation}
%--
where only two parameters, $a_2$ and $b_2$, are involved.  In
figure~\ref{relative_difference}, we plot the Yagi-Yunes
relation~(\ref{5-d_Love_Q_eq}) along with its linear fit~(\ref{2-d_Love_Q_eq}),
which is obtained by a least squares regression performed on 1000 points
uniformly picked in the logarithmic space from the Yagi-Yunes relation ($a_2=-0.1457$
and $b_2=0.3094$).  As an example of a typical EOS, we plot the
Love-Q relation calculated assuming the APR4 EOS~\cite{PhysRevC.58.1804}, a soft
EOS consistent with the observations of GW170817~\cite{LIGOScientific:2017vwq,
LIGOScientific:2018cki, LIGOScientific:2018hze}. In this case, the relative
differences in $Q$ between these two models and the APR4 EOS are smaller
than $1\%$. 

%---------------------------------------------------------------------
\begin{figure}[tbp]
\centering
\includegraphics[width=0.8\textwidth]{fig_2d-5d_difference.pdf}
% Here is how to import EPS art
\caption{Illustration of the Love-Q relation. In the upper panel, the orange
solid line indicates the original Yagi-Yunes relation~\eqref{5-d_Love_Q_eq},
while the green dashed line represents our fitting with the linear
model~\eqref{2-d_Love_Q_eq}. We also show the Love-Q relation for the APR4 EOS
as reference with blue circles.  In the lower panel, we show the absolute
relative differences between the two models (denoted as $Q_5$ and $Q_2$) and the
APR4 EOS (denoted as $Q_{\rm APR4}$). 
\LS{[LS: why figure is blurred?? Use vector figure, not pixel figure]}}
\label{relative_difference}
\end{figure}
%---------------------------------------------------------------------

%=============================
\subsection{Hierarchical Bayesian Inference}
\label{subsec:framework_principles}
%=============================

The coefficients in Eq.~\eqref{5-d_Love_Q_eq} and Eq.~\eqref{2-d_Love_Q_eq} do
not directly contribute to the GW waveform. Instead, they determine a relation
between the waveform parameters $\Lambda$ and $Q$, denoted as
$Q=f(\Lambda;\bm{H})$, where $\bm{H}$ represents the coefficients, $\bm{H} =
\{a_2, b_2\}$ or $\bm{H} = \{a_5, b_5, c_5, d_5, e_5\}$. When the universal
relation is exact, this leads to a $\delta$-function-type prior between GW
parameters $\Lambda$ and $Q$
%--
\begin{equation}
\label{delta function prior}
\pi(Q|\Lambda,\bm{H}) = \delta\big(Q-f(\Lambda;\bm{H})\big)\,.
\end{equation}
%--
Intuitively, one can measure $\Lambda$ and $Q$ from BNS GW events, and then fit
the relation with the measurements. This procedure can be implemented within the
hierarchical Bayesian framework, which is a powerful formalism in studying population
properties of GW events beyond individual observations~\cite{Thrane_2019}. 

In the population inference scenario, the population properties are
characterized by a set of hyperparameters, such as the power index of the mass
distribution. These hyperparameters do not enter the waveform directly, while
can be inferred from a collection of measurements of single-event parameters.
Similarly, the EOS parameters determine the relation between $\Lambda$ and $m$,
as well as between $Q$ and $m$, can be regarded as hyperparameters and analyzed
in the hierarchical Bayesian framework~\cite{Mandel:2009nx, Mandel:2009pc,
Adams:2012qw, Lackey:2014fwa, Mandel:2018mve, Golomb:2021tll, KAGRA:2021duu,
Wang:2024xon}. In this work, we adopt this framework to the inference of the
Love-Q relation.

Below we briefly introduce the hierarchical Bayesian framework and describe the
customization in inferring the Love-Q relation.  The hierarchical Bayesian
framework aims to find the posterior distribution of the hyperparameters
$p(\bm{H}|D)$, given the catalog-level data $D=\{d_1,\cdots,d_n\}$ consisting of
$n$ individual GW events.  Since both the hyperparameters $\bm{H}$ and the
single-event parameters $\{\bm{\theta}_1,\cdots,\bm{\theta}_n\}$ are unknown, we
write the Bayes' theorem as 
%--
\begin{equation}
\label{bayes2}
p(\bm{H},\bm{\theta}_1, \cdots,\bm{\theta}_n|D) =\frac{p(D|\bm{H},
\bm{\theta}_1, \cdots,\bm{\theta}_n)\pi(\bm{H}, \bm{\theta}_1,
\cdots,\bm{\theta}_n)}{p(D)}\,,
\end{equation}
%--
where $\pi(\bm{H},\bm{\theta}_1,\cdots,\bm{\theta}_n)$ is the prior,
$p(D|\bm{H},\bm{\theta}_1,\cdots,\bm{\theta}_n)$ is the likelihood function, and
$p(D)$ denotes the evidence. Then, $p(\bm{H}|D)$ can be obtained by 
marginalizing over the single-event parameters
%--
\begin{equation}
\label{bayes1}
p(\bm{H}|D) = \int p(\bm{H},\bm{\theta}_1, \cdots, \bm{\theta}_n|D) \text{d}
\bm{\theta}_1\cdots\text{d} \bm{\theta}_n\,.
\end{equation}
%--

Assuming that the $n$ events are independent, the prior can be decomposed into
parts of hyperparameters and that of single-event parameters,
%--
\begin{equation}
\label{bayes3}
\pi(\bm{H}, \bm{\theta}_1, \cdots,\bm{\theta}_n) = \pi(\bm{H}) \prod_{i=1}^n
\pi(\bm{\theta}_i|\bm{H})\,.
\end{equation}
%--
Since we are inferring the Love-Q relation, the tidal deformabilities, $\bm
{\Lambda}_i=\{\Lambda_{1i},\Lambda_{2i}\}$, and quadrupole moments,
$\bm{Q}_i=\{Q_ {1i},Q_{2i}\}$, of the two NSs of the $i\text{-th}$ event are
treated as independent parameters in $\bm{\theta}_i$. Note that the Love-Q
relation is insensitive to EOS, the priors of mass parameters, $m_1$ and $m_2$,
are chosen to be independent of $\bm{\Lambda}_i, \bm{Q}_i$ and $\bm{H}$. Other
parameters in $\bm{\theta}_i$ are also assumed to be independent.  In this way,
the conditional prior of $\bm{\theta}_i$ can be further decomposed as
%--
\begin{equation}
\label{prior}
\pi(\bm{\theta}_i|\bm{H}) = \pi(\bm{\Lambda}_i|\bm{H}) \,
\pi(\bm{Q}_i|\bm{\Lambda}_i,\bm{H}) \, \pi(\bm{\xi}_i)\,,
\end{equation}
%--
where $\bm{\xi}_i$---the so-called nuisance parameters hereafter---denotes the
other parameters in $\bm{\theta}_i$ except for $\bm{\Lambda}_i$ and $\bm{Q}_i$.
Analogously to the prior, the catalog likelihood can be factorized into the
product of single-event likelihoods
%--
\begin{equation}
    p(D|\bm{H},\bm{\theta}_1, \cdots,\bm{\theta}_n) = \prod_{i=1}^{n}
    p(d_i|\bm{H},\bm{\theta}_i)=\prod_{i=1}^{n}
    p(d_i|\bm{\theta}_i)\,,\label{eq:catalog_likelihood}
\end{equation}
%--
where the second equality comes from the fact that the hyperparameters do not
enter the waveform.

Given the specific form of the prior and the likelihood, the marginalized
posterior distribution \eqref{bayes1} becomes
%--
\begin{equation}
\label{hierarchical bayes}
\begin{aligned}
p(\bm{H}|D) &= \frac{1}{p(D)}\pi(\bm{H}) \int \text{d}\bm{\theta}_1
\cdots\text{d} \bm{\theta}_n \prod_{i=1}^n \big[\pi(\bm{\Lambda}_i|
\bm{H}) \, \pi(\bm{Q}_i| \bm{\Lambda}_i,
\bm{H}) \,\pi(\bm{\xi}_i) \,p(d_i|\bm{\theta}_i)\big] \\
&=\frac{1}{p(D)} \pi(\bm{H}) \prod_{i=1}^n \int \text{d}\bm{\Lambda}_i
\text{d}\bm{Q}_i \,\pi (\bm{\Lambda}_i|\bm{H}) \, \delta\big(\bm{Q}_i -
\bm{f}(\bm{\Lambda}_i; \bm{H})\big) \int \text{d} \bm{\xi}_i \,
\pi(\bm{\xi}_i)p(d_i|\bm{\theta}_i)\\
&=\frac{1}{p(D)} \pi(\bm{H}) \prod_{i=1}^n \int \text{d}\bm{\Lambda}_i \,
\pi(\bm{\Lambda}_i| \bm{H})L_i\big( \bm{\Lambda}_i,
\bm{f}(\bm{\Lambda}_i;\bm{H})\big)\,.
\end{aligned}
\end{equation}
%--
In the second line, the bold font $\bm{f}(\bm{\Lambda}_i;\bm{H})$ is used to
denote the vector form of the Love-Q relation applied to the two NSs in a BNS
system. In the third line, $L_i(\bm{\Lambda}_i,\bm{Q}_i)$, called the
quasi-likelihood, is defined as the integral over the nuisance parameters
%--
\begin{equation}
\label{quasi-likelihood}
    L_i(\bm{\Lambda}_i,\bm{Q}_i) =\int \text{d}\bm{\xi}_i \,
    \pi(\bm{\xi}_i)p(d_i|\bm{\theta}_i)\,.
\end{equation}
%--

The quasi-likelihood of each event can be computed independently of $\bm{H}$.
For the $i\text{-th}$ event, we write down the Bayes' theorem
%--
\begin{equation}
\label{single bayes}
    p(\bm{\theta}_i|d_i, \varnothing)\propto
    \pi(\bm{\theta}_i|\varnothing)p(d_i|\bm{\theta}_i)\,,
\end{equation}
%--
where $\pi(\bm{\theta}_i|\varnothing)$ denotes an auxiliary prior independent of
$\bm{H}$.  The explicit form of the single-event likelihood,
$p(d_i|\bm{\theta}_i)$, is given by assuming stationary and Gaussian
noise~\cite{Finn:1992wt}
 %--
\begin{equation}
p(d_i|\bm{\theta}_i)\propto \mathrm{e}^{ -\frac{1}{2} \langle
d_i-h(\bm{\theta}_i),d_i-h(\bm{\theta}_i)\rangle}\,,
\end{equation}
%--
with the data $d_i$ and the waveform model $h(\bm{\theta}_i)$.  The inner
product of $u(t)$ and $v(t)$, $\langle u, v\rangle$, is defined as
%--
\begin{equation}
    \langle u, v\rangle:= 2\int_{-\infty}^{\infty} \frac{\tilde{u}(f)
    \tilde{v}^{*}(f)}{S_n(|f|)} \text{d}f\,,
\end{equation}
%--
where $\tilde{u}(f)$ and $\tilde{v}(f)$ are the Fourier transforms of $u(t)$ and
$v(t)$, and $S_n(f)$ is the power spectrum density (PSD) of the noise.

Rearranging terms of Eq.~\eqref{single bayes} and substituting it into
Eq.~\eqref{quasi-likelihood}, one finds that
%--
\begin{equation}
\label{quasi-posterior}
\begin{aligned}
    L_i(\bm{\Lambda}_i,\bm{Q}_i) = \int \text{d}\bm{\xi}_i \,
    \pi(\bm{\xi}_i)p(d_i|\bm{\theta}_i) \propto \int \text{d}\bm{\xi}_i
    \frac{\pi(\bm{\xi}_i)}{\pi(\bm{\theta}_i
    |\varnothing)}p(\bm{\theta}_i|d_i,\varnothing)\,.
\end{aligned}  
\end{equation}
%--
If we further choose $\pi(\bm{\theta}_i|\varnothing) \propto\pi(\bm{\xi}_i)$, 
i.e., a flat prior for $\bm{\Lambda}_i$ and $\bm{Q}_i$,
Eq.~\eqref{quasi-posterior} can be further simplified as 
%--
\begin{equation}
\label{quasi-marginalized}
\begin{aligned}
    L_i(\bm{\Lambda}_i,\bm{Q}_i) \propto \int \text{d} \bm{\xi}_i \,
    p(\bm{\theta}_i|d_i, \varnothing)\propto p(\bm{\Lambda}_i,
    \bm{Q}_i|d_i,\varnothing)\,.
\end{aligned}  
\end{equation}
%--
This means that the quasi-likelihood is proportional to the marginalized
posterior of the auxiliary single-event inference with a flat prior on
$\bm{\Lambda}_i$ and $\bm{Q}_i$. 

The hierarchical Bayesian framework, as its name implies, introduces two levels
of inferences. The first level consists of single-event Bayesian inferences
based on Eq.~\eqref{single bayes}, where the quasi-likelihoods are constructed
according to Eq.~\eqref{quasi-marginalized}. In the second level, the
quasi-likelihoods are combined to infer the hyperparameters based on
Eq.~\eqref{hierarchical bayes}.


%=============================
\section{Simulation}
\label{sec:simulation}
%=============================

%=============================
\subsection{Waveform, Population and Detectors}
\label{subsec:simulation_preliminaries}
%=============================

In our simulation, we adopt the {\sc IMRPhenomXAS\_NRTidalv3} waveform
model~\cite{Abac:2023ujg}, which includes tidal amplitude corrections as well as
spin-induced quadrupole moment terms up to 3.5\,PN with aligned spins.  The
single-event parameters $\bm{\theta}$ include the binary masses $m_1$ and $m_2$,
the dimensionless tidal deformabilities $\Lambda_1$ and $\Lambda_2$,
spin-induced quadrupole moments $Q_1$ and $Q_2$, the dimensionless spins
$\chi_1$ and $\chi_2$, the luminosity distance $D_L$, the merge time $t_{c}$,
the right ascension $\alpha$ and declination $\delta$, the inclination angle
$\iota$, the GW polarization angle $\psi$, and the phase of coalescence
$\phi_{c}$.

When generating the BNS events, we adopt the population model proposed by
\citet{Farrow:2019xnc}.  According to the spin magnitude, the model divides a
BNS into a recycled NS and a nonrecycled (\emph{slow}) one, for which the masses
are labeled as $m_{\mathrm{r}}$ and $m_{\mathrm{s}}$, respectively.  The
distribution of $m_{\mathrm{r}}$ has two Gaussian components while
$m_{\mathrm{s}}$ follows a uniform distribution,
%---
\begin{subequations}
\label{mass population}
\begin{equation}
    P(m_{\mathrm{r}}) = \alpha \mathcal{N}(\mu_1, \sigma_1) + (1-\alpha)
    \mathcal{N}(\mu_2, \sigma_2)\,,
\end{equation}
\begin{equation}
    P(m_{\mathrm{s}}) = \mathcal{U}(m_{\mathrm{s}}^l, m_{\mathrm{s}}^u)\,,
\end{equation}
\end{subequations}
%---
where $\alpha=0.68$, $\mu_1=1.34\, \mathrm{M}_{\odot}$, $\sigma_1=0.02\,
\mathrm{M}_ {\odot}$, $\mu_2=1.47\, \mathrm{M}_{\odot}$, $\sigma_2=0.15\,
\mathrm{M}_{\odot}$, and $m_{\mathrm{s}}^l =1.14\, \mathrm{M}_{\odot}$,
$m_{\mathrm{s}}^u =1.46\, \mathrm{M}_{\odot}$. The NS with a larger mass is
labeled as the primary star with mass $m_1$ and the other as the secondary star
with mass $m_2$.  The tidal deformability and quadrupole moments of the binary
are calculated from the stellar mass assuming the APR4 EOS with methods
described in Refs.~\cite{Yagi:2013awa, Atta:2024ckt}.  For the spin of recycled
stars $\chi_{\mathrm{r}}$, we adopt a uniform distribution
$\mathcal{U}(-0.5,0.5)$, while for $\chi_{\mathrm{s}}$ we draw from
$\mathcal{U}(-0.1,0.1)$.  Using the cosmological parameters provided by the
Planck Collaboration~\cite{Planck:2018vyg}, we simulate 1000 GW sources,
corresponding to a few years' observation from XG detectors with the observed
local merger rate, $7.6$--$250\, \mathrm{Gpc}^{-3} \,
\mathrm{yr}^{-1}$~\cite{LIGOScientific:2025pvj, LIGOScientific:2020aai}.  These
sources are distributed uniformly in source-frame time and in the co-moving
volume with distance between 15~Mpc and 150~Mpc, also uniform in sky locations
and orientations. Without loss of generality, all GW events are injected with
$t_{c}=0$.

We select a XG detector network consisting of two CE detectors and one ET 
detector, whose sensitivities are taken as CE-2~\cite{Reitze:2019iox,
Reitze:2019dyk} and ET-D~\cite{Punturo:2010zz, Hild:2010id,
Sathyaprakash:2012jk}, respectively.  The two CE detectors are positioned at the
current sites of the two LIGO detectors, while the ET detector is set at the
current location of the Virgo detector with a triangular shape. 

%=============================
\subsection{Implementation}
\label{subsec:simulation_implementation}
%=============================

According to the discussion in section~\ref{subsec:framework_principles}, the
inference of the Love-Q relation can be divided into two steps.  In the
auxiliary single-event inference, the variable parameters are 
%--
\begin{align}
	\bm{\theta} = \{\mathcal{M},\eta, \Lambda_1,\Lambda_2,
	Q_1,Q_2,\chi_1,\chi_2, D_L,t_{c},\alpha, \delta,\iota,\psi,\phi_{c}\} \,.
\end{align}
%--
In the parameter estimation, the priors of $\mathcal{M}$, $\eta$, $\chi_1$,
$\chi_2$, $t_{c}$ and $\phi_{c}$ are uniform, and the prior of $D_L$ is such
that the distribution is uniform in the co-moving volume. We set isotropic
priors for the angle variables $\alpha,\delta,\iota,\psi$. For tidal and
quadrupole moment parameters, we treat $\Lambda_{\mathrm{s}}$ and
$Q_{\mathrm{s}}$ of the slow binary component as nuisance parameters, since the
spin-induced quadrupole moment is poorly estimated for NSs when the spins are
slow~\cite{Yagi:2013awa}.  According to the arguments around
Eq.~\eqref{quasi-marginalized}, we choose uniform priors for $\Lambda_1$,
$\Lambda_2$, $Q_1$ and $Q_2$ in the auxiliary inference.  To calculate the
posterior, we generate samples with the {\sc Bilby}~\cite{Ashton:2018jfp}
implementation of the {\sc nessai} sampler~\cite{Skilling:2004pqw,
Skilling:2006gxv, michael_j_williams_2025_14627250, PhysRevD.103.103006,
Williams:2023ppp}.  To calculate the integral in Eq.~\eqref{hierarchical bayes},
we need the functional form of the quasi-likelihood, which is proportional to
the distribution function of the posterior $p(\bm{\Lambda}_i,
\bm{Q}_i|d_i,\varnothing)$. We follow \citet{Golomb:2021tll} and adopt the
Gaussian mixture model to estimate the density of the posterior samples.  In the
second step, the priors of the hyperparameters, $\pi(\bm{H})$, are listed in
table~\ref{prior_table}. For the conditional prior $\pi(\bm
{\Lambda}_i|\bm{H})$, we choose the uniform distribution $\mathcal{U}(10,2000)$.

We simulate 1000 GW events as described in
section~\ref{subsec:simulation_preliminaries}.  Though the hierarchical
procedure avoids a direct high-dimensional inference for $\bm{H}$ and
$\{\bm{\theta}_1,\cdots,\bm{\theta}_n\}$ simultaneously, the computational cost
still increases with the number of events $n$.  \citet{Lackey:2014fwa} found
that when constraining the EOS of NSs, several events with the highest
signal-to-noise ratios (SNRs) contribute most of the information. Also,
\citet{Yagi:2013awa} concluded that the spin-induced quadrupole moment cannot be
well measured for NSs with low spins. Considering these findings, we first draw
100 sources with the highest SNRs, then further select 20 sources with the
largest $|\chi_{\mathrm{r}}|$ among them. As we will show in
section~\ref{sec:results}, the primary information for constraining the Love-Q
relation comes from the 10 loudest GW events. We leave a more comprehensive
study of the whole population for future work. 

%---------------------------------------------------------------------
\begin{table}[t]
    \centering
    \sisetup{
        table-align-text-post = false, 
        separate-uncertainty = true 
    }
        \caption{The reference values and priors of the coefficients in
        different models of the Love-Q relation, with $j$ being the number of
        parameters in the polynomial model. Note that $j=5$ indicates the
        original Yagi-Yunes relation~\cite{Yagi_2017}. For $j = $ 2, 3 and 4,
        the values are the results of least squares regression performed on 1000
        points uniformly picked in the logarithmic space from the Yagi-Yunes
        relation. Priors are chosen to be uniform distributions, and the ranges
        are the same for different $j$ to ensure a meaningful comparison.
        }\label{prior_table}
    \begin{tabular}{
        l
        S[table-format=-1.4]
        S[table-format=1.5]
        S[table-format=1.3e-1]
        S[table-format=-1.3e-1]
        S[table-format=1.3e-1]
    }
        \toprule
        \multicolumn{6}{c}{Reference Values} \\
        %\cmidrule(l){2-6}
        $j$ & {$a_j$} & {$b_j$} & {$c_j$} & {$d_j$} & {$e_j$} \\
        \midrule

        5 \, &  0.1940 & 0.0916 & 4.812e-2 & -4.283e-3 & 1.245e-4 \\
        4 \, &  0.1290 & 0.1480  & 3.021e-2 & -1.817e-3 & {--}      \\
        3 \, & -0.0709 & 0.2775  & 3.220e-3 & {--}       & {--}      \\
        2 \, & -0.1457 & 0.3094  & {--}      & {--}       & {--}      \\
        
        \midrule

        \multicolumn{6}{c}{Priors} \\
        %\cmidrule(l){2-6}
        $j$ & {$a_j$} & {$b_j$} & {$c_j$} & {$d_j$} & {$e_j$} \\
        \midrule

        5 \, & {$\mathcal{U}(-5.0, 5.0)$} & {$\mathcal{U}(-1.0, 1.0)$} & {$\mathcal{U}(-0.5, 0.5)$} & {$\mathcal{U}(-0.1, 0.1)$} & {$\mathcal{U}(-0.01, 0.01)$} \\
        4 \, & {$\mathcal{U}(-5.0, 5.0)$} & {$\mathcal{U}(-1.0, 1.0)$} & {$\mathcal{U}(-0.5, 0.5)$} & {$\mathcal{U}(-0.1, 0.1)$} & {--}                         \\
        3 \, & {$\mathcal{U}(-5.0, 5.0)$} & {$\mathcal{U}(-1.0, 1.0)$} & {$\mathcal{U}(-0.5, 0.5)$} & {--}  & {--}   \\
        2 \, & {$\mathcal{U}(-5.0, 5.0)$} & {$\mathcal{U}(-1.0, 1.0)$} & {--}  & {--}   & {--}  \\
        \bottomrule
    \end{tabular}
\end{table}
%---------------------------------------------------------------------


%=============================
\section{Results and Discussions}
\label{sec:results}
%=============================

In this section, we present the inference results for different parameterization
models of the Love-Q relation. In section~\ref{subsec:results_linear_model}, we 
show the results of the linear model, Eq.~\eqref{2-d_Love_Q_eq}, consisting of
two parameters.  In section~\ref{subsec:results_quartic}, we present the results
of the quartic polynomial model in Eq.~\eqref{5-d_Love_Q_eq}, consisting of five
parameters. The polynomial models in between, i.e., the quadratic and cubic
polynomial models, are discussed in
section~\ref{subsec:results_quadratic_cubic}.

%=============================
\subsection{The Linear Model}
\label{subsec:results_linear_model}
%=============================

%---------------------------------------------------------------------
\begin{figure}[t]
    \centering
    \includegraphics[width=0.5\linewidth]{fig_comparison_corner_plot.pdf}
    \caption{Posterior distributions of the hyperparameters ${\bm H} =
    \{a_2,b_2\}$ in the linear fitting model. The contours refer to 50\% and
    90\% credible regions, while the numbers above the histograms on the
    diagonal stand for the median and the central 50\% \LS{[LS: maybe we can
    change them to 90\%?]} credible interval of the marginalized distribution.
    We use blue and red colors to represent the results based on the loudest 20
    and 10 events from the 1000 simulated events, respectively.  The black lines
    represent the reference values in table~\ref{prior_table}.  The grey lines
    on the diagonal represent the priors for comparison.}
    \label{corner2-d}
\end{figure}
%---------------------------------------------------------------------

We first perform the inference for the linear model, which is also the model 
adopted in Ref.~\cite{Samajdar:2020xrd}. The posterior distribution of the 
hyperparameters $\{a_2,b_2\}$ is shown in figure~\ref{corner2-d}. Compared to
the priors in table~\ref{prior_table}, the posteriors are significantly 
narrowed. As a comparison, we also mark the fitting values of the
hyperparameters in a direct fit of the Yagi-Yunes Love-Q relation, and regard
them as ``true'' values of the linear model for reference. These values fall
within the 90\% credible region of the posterior. The hierarchical Bayesian
inference successfully recovers the Love-Q relation under the linear
parameterization. We also investigate how the results depend on the number of
events in the analysis. In the 20 events selected in 
section~\ref{subsec:simulation_implementation}, we further select the loudest 10
events, perform the inference again, and show the results in figure~\ref
{corner2-d}. For both the joint and marginalized distributions, the widths of
the credible regions in the 10-event inference are only slightly larger than
those in the 20-event inference. This reveals that the loudest 10 events
dominate the information in constraining the Love-Q relation, and including
quieter events will not significantly change the results.

%---------------------------------------------------------------------
\begin{figure}[t]
\centering
\includegraphics[width=\textwidth]{fig_hierarchical_results_APR4_2d.pdf}
    \caption{Recovered Love-Q relation from the posterior of the hyperparameters
    in the linear model. In the left panel, the Love-Q relation is inferred with
    the 20 loudest events from the 1000 simulated GW events. The gray points
    mark the median values of inferred $\Lambda$ and $Q$ for each event with
    $68\%$ errorbars. The blue solid line marks the median of the distribution
    of $Q$ as a function of $\Lambda$, accompanied by the $50\%$ and $90\%$
    credible intervals in shaded regions.  The red dashed line represents the
    maximum-posterior Love-Q relation. For comparison, we plot the original
    Yagi-Yunes Love-Q relation~\cite{Yagi_2017} in orange. The right panel shows
    how the $90\%$ credible region of the recovered Love-Q relation depends on
    the number of events, marked with different colors. \LS{[LS: use larger
    fonts for labels and legends; figure 1 looks good]} \LS{[LS: in the right
    panel, change ``Yagi-Yunes Relation'' to ``Yagi-Yunes relation'' (same as
    left panel), and add ``events'' after 5, 10, 15, 20, and remove ``90\%'']}
    \LS{[LS: add $\Lambda$ to both panels' $x$-axis]} }    \label{2-d_Love_Q}
\end{figure}
%---------------------------------------------------------------------

In figure~\ref{2-d_Love_Q}, we show the recovered Love-Q relation according to
the posterior samples. For each $\Lambda$, every sample in the posterior of the 
hyperparameters corresponds to a $Q$ value. For a fixed $\Lambda$ value, we find
the credible intervals of $Q$, and then vary $\Lambda$ continuously to form
credible regions. In the left panel, we show the results of the 20-event
inference. Similar to figure~\ref{corner2-d}, the Yagi-Yunes Love-Q relation is
covered by the 90\% credible region.  In the right panel, we select the loudest
5, 10, 15 and 20 events from the 20 events to test how the recovered Love-Q
relation depends on the number of events. We find that the widths of the 90\%
credible regions are almost the same for the inferences from 10, 15 and 20
events. This is consistent with the posteriors in figure~\ref{corner2-d}, and
again indicates that the loudest 10 events are almost sufficient to constrain
the Love-Q relation. Similar phenomena were also found in previous
studies~\cite{Lackey:2014fwa, Landry:2020vaw, Pang:2020ilf, Finstad:2022oni,
Bandopadhyay:2024zrr, Wang:2024xon}, where the recovered $\Lambda$-$m$ relation
is dominated by the several loudest events.

%=============================
\subsection{The Quartic Polynomial Model}
\label{subsec:results_quartic}
%=============================

%---------------------------------------------------------------------
\begin{figure}
\begin{minipage}[t]{0.49\textwidth}
\centering
\includegraphics[width=0.8\linewidth]{fig_Hyper_parameter_5d.pdf}
% Here is how to import EPS art
\end{minipage}
\hfill
\begin{minipage}[t]{0.49\textwidth}
\includegraphics[width=\linewidth]{fig_hierarchical_results_APR4_5d.pdf}
\end{minipage}
    \caption{The posterior of the hyperparameters and the recovered Love-Q
    relation in the quartic polynomial model, where all 20 loudest events are
    included in the inference. Figure settings of two panels are respectively
    similar to those in figure~\ref{corner2-d} and the left panel of
    figure~\ref{2-d_Love_Q}. \LS{[LS: use larger fonts; use ``$10^3 d_5$''
    instead of ``$d_5/10^{-3}$'', same for $e_5$]}
    } \label{5-d_Love_Q} 
\end{figure}
%---------------------------------------------------------------------

For the quartic polynomial model, the Love-Q relation is fitted with five 
parameters shown in Eq.~\eqref{5-d_Love_Q_eq}, i.e., $\{a_5, b_5, c_5, d_5,
e_5\}$.  This is  the original model proposed by Yagi and
Yunes~\cite{Yagi:2013awa}. We summarize the posterior and the recovered Love-Q
relation in figure~\ref{5-d_Love_Q}, where all 20 events are included in the
inference. In the left panel, though the true values (the values in the
Yagi-Yunes Love-Q relation~\cite{Yagi:2013awa}) are almost centered in the
distribution, the posteriors are much wider than those in the linear case,
reaching the prior boundaries. This indicates that the $\Lambda$ and $Q$
measurements from these events are not informative enough to well constrain all
the five parameters. In other words, the observation precision is not high
enough to capture the higher-order terms introduced by the additional three
parameters, $c_5, d_5$, and $e_5$. This is also reflected in the strong
correlations between the three parameters shown in the left panel of
figure~\ref{5-d_Love_Q}. In the right panel of figure~\ref{5-d_Love_Q}, the 90\%
credible regions of the recovered Love-Q relation are also wider than that in
the linear case, especially for large $\Lambda$ where the higher-order terms
become more important. For $\Lambda \sim 400$, the widths of the 90\% credible
region between the two models are similar, which is consistent with the fact
that most of the simulated events gather around this region. 
% The Yagi-Yunes Love-Q relation is mostly covered by the 50\% credible region,
% and well within the 90\% credible region.

%=============================
\subsection{The Quadratic and Cubic Polynomial Models}
\label{subsec:results_quadratic_cubic}
%=============================

%---------------------------------------------------------------------
\begin{figure}[t]
    \begin{minipage}[t]{0.49\textwidth}
    \centering
    \includegraphics[width=0.8\linewidth]{fig_Hyper_parameter_3d.pdf}
    \end{minipage}
    \hfill
    \begin{minipage}[t]{0.49\textwidth}
    \centering
    \includegraphics[width=0.8\linewidth]{fig_Hyper_parameter_4d.pdf}
    \end{minipage}
    \vspace{3mm}
    \begin{minipage}[t]{\textwidth}
    \includegraphics[width=\linewidth]{fig_hierarchical_results_APR4_3d.pdf}
    \end{minipage}
    \caption{Similar to figure~\ref{5-d_Love_Q}, while for the quadratic and
    cubic polynomial models. \LS{[LS: use larger fonts; change ``$d_4/10^{-3}$''
    into ``$10^3 d_4$''; maybe use the same layout as in Figure 4, namely
    posterior on the left and $Q$-$\Lambda$ relation on the right?]}
    }\label{3-d_4-d_Love_Q}
\end{figure}
%---------------------------------------------------------------------

As shown in the previous two subsections, the observations from 20 simulated GW 
events can well constrain the two parameters in the linear model, while cannot 
effectively constrain the five parameters in the quartic polynomial model. The
above two models were adopted in previous studies~\cite{Yagi:2013awa,
Samajdar:2020xrd}, and here we further test the models in between them and see
how the constraints change with the number of hyperparameters. We perform the
inference for the quadratic and cubic polynomial models, which contain three
parameters, $\{a_3, b_3, c_3\}$, and four parameters, $\{a_4, b_4, c_4, d_4\}$,
respectively. Similar to the linear model, the reference values of these
parameters are obtained by directly fitting the Yagi-Yunes Love-Q relation and
are listed in table~\ref{prior_table}. The priors of the hyperparameters are
kept the same as their counterparts in the quartic polynomial model, which are
also listed in table~\ref{prior_table}.

We summarize the posterior distributions of the hyperparameters in the top row
of figure~\ref{3-d_4-d_Love_Q}. In the quadratic model, strong degeneracy arises
among the three parameters. The posterior of $b_3$ is almost the same as its 
prior. For $a_3$ and $c_3$, though the posteriors are narrower than the priors, 
their marginal distributions have wide and flat plateau. In the cubic model, 
strong degeneracy  exists between $c_4$ and $d_4$, and the posterior of $b_4$ is
similar to its prior. In general, posterior with large correlations and 
prior-like marginal distributions indicates redundant parameters in the model. 
Therefore, we conclude that the linear model is accurate enough in constraining 
the Love-Q relation with XG GW observations. This is consistent with the
argument in Ref.~\cite{Samajdar:2020xrd} based on qualitative analysis, whereas
in this work we present a quantitative demonstration. 

Same as before, we plot recovered Love-Q relations in the lower panels of 
figure~\ref{3-d_4-d_Love_Q}. We find that the widths of the 90\% credible
regions around $\Lambda \sim 400$ are similar for all of four models. While for
$\Lambda$ away from this region ($\Lambda \lesssim 200$ or $\Lambda \gtrsim
500$), the widths increase with the number of parameters. This may have resulted
from the increased model complexity and the lack of data points in these
regions. 

%=============================
\section{Testing Modified Gravity: Dynamical Chern-Simons Gravity}
\label{sec:dCS}
%=============================

\begin{figure}[t]
    \centering
    %\begin{minipage}{0.48\linewidth}
    %    \includegraphics[width=\linewidth]{CS_zeta_APR4_2d.pdf}
    %\end{minipage}
    %\hfill
    %\begin{minipage}{0.48\linewidth}
    %    \includegraphics[width=\linewidth]{CS_xi_cs_APR4_2d.pdf}
    %\end{minipage}
    %\vspace{3mm}
    \begin{minipage}{0.6\linewidth}
        \includegraphics[width=\linewidth]{fig_CS_xi_bar_APR4_2d.pdf}
    \end{minipage}
    \caption{Comparison of recovered Love-Q relation in GR
    and dCS predictions 
    with coupling constant $\bar{\xi}$ fixed. Similar to figure~\ref{2-d_Love_Q}, the
    blue lines indicate the median of the recovered Love-Q relation, while the
    shaded regions represent the $50\%$ and $90\%$ credible intervals. For the coupling constant 
    $\bar\xi$, we take three values
    and plot the corresponding Love-Q relation.}
    \label{cs_Love_Q}
\end{figure}
In this section, we investigate the potential of testing gravity theories with the 
inferred Love-Q relation from the GW observations. The I-Love-Q test can be 
powerful when significant difference in I-Love-Q relation between GR and modified 
gravity theories exists, for example, some parity-violating theories~\cite{Yagi_2017,Yunes:2025xwp}. 
Here we take the dynamical Chern-Simons (dCS) gravity~\cite{Jackiw:2003pm,Smith:2007jm,Alexander:2009tp} for discussion, 
which is well-motivated from heterotic superstring theory~\cite{Polchinski:1998rq,
Polchinski:1998rr}, loop quantum gravity~\cite{Alexander:2004xd,Taveras:2008yf,
Calcagni:2009xz} and effective field theories of inflation~\cite{Weinberg:2008hq}. 
The dCS gravity introduces parity violation and quadratic curvature terms into the action~\cite{Alexander:2009tp,Gupta:2017vsl}
\begin{equation}
   \label{cs_action}
   S = \int \mathrm{d}^4 x \sqrt{-g}\left[ \kappa_g \mathcal{R} + \frac{\alpha}{4} \mathcal{\vartheta} \mathcal{R}_{\nu\mu\rho\sigma} {}^{*}\mathcal{R}^{\mu\nu\rho\sigma} - \frac{\beta}{2}\nabla_{\mu}\mathcal{\vartheta}\nabla^{\mu}\mathcal{\vartheta} + \mathcal{L}_{\mathrm{mat}}\right]\,,
\end{equation}
where $g$ is the determinant of the metric, $\kappa_g= 1/16\pi$, $\mathcal{R}$ is 
the Ricci scalar, $\mathcal{R}_{\nu\mu\rho\sigma}$ and
$^{*}\mathcal{R}^{\mu\nu\rho\sigma}$ denote the Riemann tensor and its dual, 
$\mathcal{L}_{\mathrm{mat}}$ is the matter Lagrangian density, 
$\alpha$ and $\beta$ are the coupling constants in dCS gravity and the
potential for the pseudo-scalar field $\mathcal{\vartheta}$ in the is omitted. $\mathcal{\vartheta}$ and $\beta$ is taken to be 
dimensionless, so $\alpha$ have the dimension of $(\mathrm{length})^2$ 
and the quantity $\xi_{\mathrm{CS}}^{1/4} \equiv [\alpha^2/
(\kappa\beta)]^{1/4}$ can be explained as 
the characteristic length of dCS gravity~\cite{Yunes:2009hc,Yagi:2012ya}. 
Current Solar System observations have constrained the characteristic length to 
$\xi_{\mathrm{CS}}^{1/4}<\mathcal{O}(10^8)$~km~\cite{Ali-Haimoud:2011zme,Yagi:2012ya}. 

%\ZW{Maybe you need to re-introduce the parameters in the Love-Q relation and the relation between them and the coupling constant in dCS gravity.}
The dCS gravity predicts Love-Q relations that deviate from the one in GR~\cite{Yagi:2013bca,Yagi:2013awa,Gupta:2017vsl}, 
allowing us to test dCS gravity with the inferred Love-Q relation. 
For a Love-Q test, Ref.~\cite{Yagi:2013mbt} obtained the dCS correction to the NS 
quadrupole moment and Ref.~\cite{Yagi:2011xp} indicates that the tidal 
deformability is the same as the one in GR at leading order in small coupling 
approximation $\zeta \equiv \xi_{\mathrm{CS}} m^2/R^6 \ll 1$, regarding the dCS 
gravity as an effective theory. Refs.~\cite{Yagi_2017,Yagi:2013mbt,Gupta:2017vsl} 
have discussed the Love-Q relation under dCS gravity and find that the relation 
becomes EOS-sensitive with $\xi_{\mathrm{CS}}$ or $\zeta$ fixed. However, with 
$\bar{\xi}\equiv \xi_{\mathrm{CS}}/m^4$ fixed, the Love-Q relation remains 
universal and the EOS variation is of $\mathcal{O}(1\%)$. Ref.~\cite{Gupta:2017vsl} 
parameterized the relation between the dCS correction to $Q$ (denoted as $Q_{\mathrm{CS}}$) and the tidal deformability $\Lambda$ as follows
\begin{equation}
    \label{cs_Love_Q_eq}
    \ln (Q_{\mathrm{CS}}/\bar{\xi}) = a + b \ln \Lambda + c \ln^{2} \Lambda
\end{equation} 
where the fitting coefficients $a=-3.443, b=-0.550$ and $c=-0.023$. 

We compare the inferred constraints of the Love-Q relation 
using the linear model (figure~\ref{2-d_Love_Q}) and Love-Q 
relations with $\bar{\xi}$ fixed (Eq.~\ref{cs_Love_Q_eq}) in dCS gravity. From 
figure~\ref{cs_Love_Q} we can conclude that our Love-Q test of dCS gravity can 
constrain the coupling constant $\bar{\xi} \lesssim 10^{3}$. Substituting typical 
mass $m=1.4~\mathrm{M_{\odot}}$ and radius $R=10$~km of NSs, we have constraints 
for the other two coupling constants $\xi_{\mathrm{CS}}^{1/4} \lesssim 10$~km and 
$\zeta \lesssim 0.1$, which are in agreement with the results 
given by Refs.~\cite{Yagi:2013bca,Yagi:2013awa}. 

%=============================
\section{Conclusion}
\label{sec:conclusion}
%=============================

In this work, we investigate the prospects of inferring the Love-Q relation of NSs
with future ground-based GW observations.
Extending the inference procedure in
Ref.~\cite{Samajdar:2020xrd}, 
We adopt the hierarchical Bayesian framework to effectively combine the
information from multiple GW events, extending the linear fitting model in
previous works~\cite{Samajdar:2020xrd}. The hierarchical Bayesian framework
separate the inferences into two steps, the auxiliary single-event
inference and the hyperparameter inference, which avoids a direct high-dimensional
inference for all the parameters simultaneously while still takes into account
degeneracy and non-Gaussianity in the single-event parameters. We also discuss
the impact of the event number on the constraints, and find that the loudest 10 events
dominate the information in constraining the Love-Q relation. Similar phenomena
were also found in studies of constraining the EOSs with GW observations~\cite{Lackey:2014fwa,Landry:2020vaw,Pang:2020ilf,Finstad:2022oni,Bandopadhyay:2024zrr,Wang:2024xon}.

We are the first to conduct a systematic study on different parameterization models
when inferring the Love-Q relation with GWs. Considering the quartic polynomial
model proposed by \citet{Yagi:2013awa} and the linear model adopted by
\citet{Samajdar:2020xrd} as two extremes, we conduct the inference with
four polynomial models
from linear to quartic terms respectively. As shown in 
section~\ref{sec:results}, we quantitatively demonstrate that 
the linear model is accurate enough to describe the Love-Q relation in the 
inference. Also, as the number of parameters increases, more significant 
degeneracy or poorly constrained parameters appear in the posteriors, while the recovered Love-Q relations keep similar in the region where most data points gather.

We also test the potential of using the inferred Love-Q relation to
constrain modified gravity theories in Section~\ref{sec:dCS}.
Taking the dCS gravity as an example, we find that the inferred Love-Q relation
can place a constraint on the dCS characteristic length about
$\xi_{\mathrm{CS}}^{1/4} \lesssim 10$~km, which is seven orders
of magnitude tighter than that from current Solar System
observations~\cite{Ali-Haimoud:2011zme,Yagi:2012ya}, and consistent with
previous predictions~\cite{Yagi:2013bca,Yagi:2013awa}. This highlights the power
of inferring the Love-Q relation from GWs in testing gravity theories. 

There are several extensions to this work. Firstly, we assume the aligned-spin BNS systems, while
realistic BNS systems may have spin precession, which leads to additional
biases or uncertainties~\cite{Williamson:2017evr}. The waveform modeling can
also have similar issues~\cite{Purrer:2019jcp,Gamba:2020wgg}. 
Also, in the XG era, the GW signals are possibly overlapping with each
other, complicating the data analysis~\cite{Pizzati:2021apa, Samajdar:2021egv,
Wang:2023ldq, Johnson:2024foj, Wang:2025ckw}.
Moreover, the Yagi-Yunes universal relations were originally found and discussed
for a single NS. For BNS systems, these universal relations may hold when the
two NSs have similar masses, but still require careful
inspection~\cite{Shao:2022koz, Saffer:2021gak}. Additionally, the octupole or
higher-order multipole moments of NSs also contribute to the GW waveform and can
be universally related to the quadrupole moment~\cite{Yagi_2017,Abac:2023ujg}.
We leave the exploration of simultaneous or independent inference of more
universal relations to future works.

%=============================
\acknowledgments

This work was supported by the Beijing Natural Science Foundation (QY25102,
1242018), the National Natural Science Foundation of China (123B2043, 12573042),
the National SKA Program of China (2020SKA0120300), the Max Planck Partner Group
Program funded by the Max Planck Society, and the High-Performance Computing
Platform of Peking University. 


% \clearpage

\bibliographystyle{apsrev4-1}
% \bibliographystyle{unsrt}
\bibliography{HBAGW_jcap}
\end{document}
