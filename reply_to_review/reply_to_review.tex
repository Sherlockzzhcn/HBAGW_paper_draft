\documentclass[11pt]{article}
\usepackage[utf8]{inputenc}
\usepackage[margin=1in]{geometry}
\usepackage{xcolor}
\usepackage{titlesec}
\usepackage{enumitem}
\usepackage{amsmath, amsthm, amssymb, appendix, bm, graphicx, hyperref, mathrsfs, tabularx}
\usepackage{mdframed}
\usepackage{subcaption, float}
\usepackage{caption}
\captionsetup[figure]{labelfont=bf, labelsep=colon, font=small}
\newcommand{\ZW}[1]{\textcolor{magenta}{$\mathcal{ZW}$:~#1}}

\definecolor{commentgray}{rgb}{0.95,0.95,0.95}
\definecolor{revcolor}{rgb}{0.0, 0.4, 0.7} 
\definecolor{oldcolor}{rgb}{0.8, 0.0, 0.0}

\newenvironment{reviewercomment}[1]{
    \vspace{1em}
    \noindent \textbf{#1:} \itshape
}{
    \par\vspace{0.5em}
}

\newenvironment{authorresponse}{
    \noindent \textbf{Reply:} \normalfont
}{
    \par\vspace{1.5em}
}

\newenvironment{originaltext}{
    \begin{quote}
    \color{oldcolor} \small \textbf{Original text:} \\
}{
    \end{quote}
}

\newenvironment{revisedtext}{
    \begin{quote}
    \color{revcolor} \small \textbf{Revised text:} \\
}{
    \end{quote}
}

\title{Replies to Editor Report}
\author{Manuscript ID: JCAP\_160P\_1025}
\date{}

\begin{document}

% —— 横线 1:标题上方(可选)
\noindent\rule{\textwidth}{0.8pt}

\begin{center}
{\LARGE\bfseries Replies to Editor Report\par}
\vspace{0.2cm}
{Manuscript ID: JCAP\_160P\_1025\par}
\end{center}
\vspace{-0.2cm}
\noindent\rule{\textwidth}{0.8pt}

\vspace{0.2cm}
\noindent Dear Editor and Reviewers,

We would like to thank the reviewers for their thoughtful comments and
constructive suggestions, which have helped us improve our manuscript.
For clarity, the {\color{oldcolor} original text} is in {\color{oldcolor} red}
and the {\color{revcolor} revised content} in {\color{revcolor} blue}.

\noindent\rule{\textwidth}{0.8pt}

% 正文从这里开始
\section*{Major questions}
\begin{reviewercomment}{1}
    The authors picked the 20 events with the largest SNRs (and spins) out of the 1000 sources they simulated. Wouldn't this choice introduce biases or selection effects in the results? 
\end{reviewercomment}


\begin{authorresponse}
    
    1. Biases. 
    
    We thank the reviewer for pointing out this important issue. As shown in
    Fig.~\ref{fig:a_b_n}, indeed there are biases for limited number of
    events. For small $N_{\text{events}}$ (like $N=5$), there is indeed a
    noticeable offset between the median estimate and the reference value 
    of the hyperparameters.
    However, as $N_{\text{events}}$ increases, the median estimate is
    approaching the reference value. This implies that while biases do exist,
    the methodology is robust as the sample size grows. The choice of 20 loudest
    events can be justified with Fig.~\ref{fig:width_n}. We can observe that
    the constraint to Love-Q relation at $\Lambda \sim 350$ (the ``waist'' part
    in Fig.~3 of the manuscript) converge to a stable value. 
\ZW{It may be better if
     the legends and ticklabels 
      in the figures are bigger.}
    2. Selection effects.

    In the gravitational-wave (GW) context, detection is usually well
    approximated as a cut on the signal-to-noise ratio (SNR), and
    selection effect is characterized by detection
    probability~\cite{LIGOScientific:2016, Mandel2018, Thrane2019}. It is
    demonstrated in Fig.~\ref{fig:snr_q_spin} that the SNR of one GW event is
    insensitive to either $Q$ (or Love-Q relation) or the spin component.
    Changing the spins would also not lead to a significant change of SNR and
    thus the
    detection probability.\ZW{Do we have some numerical results to support this statement?} Therefore,
    our
    inference of Love-Q relation would not be significantly affected by
    selection effect. In practice, we just discard those events with small spins
    from which no available information on $Q$ can be drawn. \ZW{It seems that
    we do not include this point in the manuscript. To reflect our response to
    the reviewer's comment, it would be better to add a sentence in the manuscript to clarify
    this point, like ``However, we agree with the reviewer that including the
    selection effect in the inference is a more rigorous approach, 
   and we leave it for future work.   
   We add a sentence on XXX (location in the manuscript) to clarify this point
   as''}
    \begin{revisedtext}
        ``In this work, we omit the selection effects since different $\Lambda$ and $Q$ values do not lead to a significant change of SNR and thus the detection probability.''
    \end{revisedtext}
\ZW{You can revise the above sentence for better readability.}
\end{authorresponse}

\begin{figure}[H]
    \centering
    \includegraphics[width=\textwidth]{2d\_model\_n\_changing.pdf}
    \caption{The dependency of the 5th, 50th and 95th percentiles of the hyperparameter posterior samples on the number of events. The orange dashed line represents the reference values, i.e. the linear fitting values obtained in section 2.1.}
    \label{fig:a_b_n}
\end{figure}

\begin{figure}[H]
    \centering
    \includegraphics[width=0.6\textwidth]{width\_n\_changing.pdf}
    \caption{The dependency of 5th and 95th percentiles of $Q$ samples computed at $\Lambda=350$, where most of the data points gather, on the number of events. The orange dashed line represents $Q$ calculated from Yagi-Yunes relation.}
    \label{fig:width_n}
\end{figure}

\begin{figure}[H]
    \centering
    \includegraphics[width=\textwidth]{snr\_Q\_spin.pdf}
    \caption{SNR under various parameter combinations. Without loss of generality, we assume a equal mass binary, and two typical values 1.4 and 1.6 $M_{\odot}$ are taken. In the left panel, $Q$ varies for fixed $\Lambda$. Therefore this case can be regarded as the Love-Q relation varies. In the right panel, only the spin of one component is changed.}
    \label{fig:snr_q_spin}
\end{figure}

\begin{reviewercomment}{2}
    Particularly, as a follow-up to 1, what would happen to the posteriors presented, for example, in Figs.~3-4 if more than 20 events are combined? Would the reference values (i.e. the “true” values from the Yagi-Yunes relations) lie outside the 90\% credible level should the constraints get narrower from including more events? Additionally, it would be great if the authors can clarify why the Yagi-Yunes reference values were taken as the true values in these corner plots. 
\end{reviewercomment}

\begin{figure}[t]
    \centering
    \includegraphics[width=0.5\textwidth]{comparison\_corner\_plot.pdf}
    \caption{Revised corner plots accounting for the fitting uncertainties of Yagi-Yunes relation. The orange bands represent the 90\% credible interval of the fitting coefficients for various EOSs.}
    \label{fig:corner}
\end{figure}
\begin{authorresponse}

    For the number of considered events, we have shown the results for more than 20 events in Fig.~\ref{fig:a_b_n}
    and \ref{fig:width_n}. While the credible interval boundary adjacent to the
    reference value remains relatively stable, the opposite boundary exhibits a
    clear contraction toward the reference value as $N_{\text{events}}$
    increases. More importantly, the median estimate approaches the reference value as $N_{\text{events}}$ increases. This suggests that the reference values would not lie outside the 90\% credible level when more events are included.
     
    For the choice of reference values, in the linear model case, we have written
\begin{originaltext}
    ``As a comparison, we also mark the fitting values of the
hyperparameters in a direct fit of the Yagi-Yunes Love-Q relation, and regard
them as ``true'' values of the linear model for reference.''
\end{originaltext}
We admit that the definition of ``true'' values is somewhat ambiguous, and in
the following we refer to this choice as reference values. Also, the fitting
itself have uncertainties with $\mathcal{O} (1\%)$ relative differences, which
represent the errors between the model and the EOS data. Therefore,
we've revised our corner plots to account for the fitting uncertainties
(Fig.~\ref{fig:corner}). The reference values are now represented by a band
rather than a single line. For the quatric model ($j=5$), we adopt the same
procedure as to obtain the reference values and uncertainties. These values are
slightly different from the original Yagi-Yunes values (which also adopted a
quartic model) since the fitting is performed with different data points and
different EOSs, but the relative differences are still $\mathcal{O}
(1\%)$\ZW{Plz check}. 
As a short summary, the ``true'' values in the corner plots are the fitting
values in our direct fit, instead of the original Yagi-Yunes values. However,
as the main goal is to constrain the Love-Q relation, in Figs.~3-4 of the
manuscript we also plot the original Yagi-Yunes relation as a reference. 

Considering the reviewer's comment, we have revised the corner plots and
corresponding descriptions in the manuscript to clarify the choice of reference
values and to account for the fitting uncertainties. For example, when
describing Fig.~2, we write \ZW{Plz add corresponding descriptions in the
manuscript and here. Both marked in blue.}
\end{authorresponse}



\begin{reviewercomment}{3}
    Regarding the Yagi-Yunes relations, the I-Love-Q universal relations introduced in Yagi \& Yunes are calculated on the assumption of slow rotation --- it has been shown that for moderate spins ($|\chi|\gtrsim 0.1$) that these universal relations break down (i.e. Pani et al 2015, Doneva et al 2014) and can deviate significantly. Since the authors consider spins up to 0.5, beyond where the universal relation is expected to break down (and much larger than observed spins from within a BNS to date), can the authors comment on how this breakdown of the universal relation would impact their results with appropriate references? In the last chapter the authors consider deviations from Love-Q as a signature of beyond GR theories, but could other assumptions such as not including rotation not also cause deviations that would be confused with this?
\end{reviewercomment}

\begin{authorresponse}
    We agree with the reviewer that high spin values ($ |\chi| \gtrsim 0.1$) do
    introduce deviations from Yagi-Yunes relations which are derived under
    slow-rotation condition. For rapidly rotating neutron stars,
    Ref.~\cite{Pani:2015nua} reveals that the I-Love relation becomes more
    sensitive to EOS. Such cases where no universal relations exist are the
    boundary of this work. However, there are still some efforts to find new
    universal relations for rapidly rotating neutron stars. With fitting
    coefficients depending on the spin parameter,
    Refs.~\cite{Doneva:2013rha,Pappas:2013naa,Doneva:2014faa} have developed new
    relations insensitive to certain set of EOSs in rapidly-rotating case.
    Ref.~\cite{Chakrabarti:2013tca} further suggested introducing the fitting
    coefficients as a function of both spin and radius, extending the
    universality to various EOSs. In these cases, the hierarchical inference
    still applies with appropriate model parameterizations and modifications, and our work provides a useful reference.
    
    We acknowledge that high-spin breakdown could potentially be confused with
    non-GR signatures. The current discussion uses the slow-rotation
    approximation as a simplified assumption and aims to providing a methodology
    reference and an estimation for order of magnitude. \ZW{Similarly, Plz add
    these refs and discussions concisely in the manuscript, maybe in both
    introducing the calculation, and possible extensions in the discussion.}
\end{authorresponse}

\begin{reviewercomment}{4}
    Can the authors explicitly write in the text how $Q$ was calculated for the parameter estimation injections? It would appear from the references it is using the slow rotation approximation, but clarification would be appropriate. 
\end{reviewercomment}

\begin{authorresponse}
    We agree with the reviewer that a clarification is better for readers. In
    this work, $Q$ was calculated from the neutron star mass given the APR4
    equation of state. We've revised our manuscript from 
    \begin{originaltext}
        ``The tidal deformability and quadrupole moments of the binary
        are calculated from the stellar mass assuming the APR4 EOS with methods
        described in Refs. [39,78].'' 
    \end{originaltext}
    to 
    \begin{revisedtext}
        ``The tidal deformability and quadrupole moments of the binary
        are calculated from the stellar mass assuming the APR4 EOS with methods
        described in Refs. [39,78] under the slow rotation approximation.''
    \end{revisedtext}
\end{authorresponse}


\begin{reviewercomment}{5}
    The authors mentioned that the APR4 equation of state (EOS) was chosen due to its consistency with GW170817. Other EOSs were also consistent with GW170817, such as SLy. Do the results of this study change with different EOS? Furthermore, what happens if multiple EOSs are assumed? Does the linear Love-Q relation inferred in this study hold for a spread of EOSs?
\end{reviewercomment}

\begin{authorresponse}
    Different assumptions of EOS just provide different combinations of $\Lambda$ and $Q$, which still follow the universal relation. As shown in Fig.~\ref{fig:corner}, the uncertainties of fitting coefficients brought by multiple EOSs assumed are much smaller compared to those from the hierarchical inference. Thus the change of EOS would not introduce a significant change to the width of constraints, while the median may probably change with EOS selected. 

    As a supplement, we have repeated the inference of linear model for SLy and
    the results are shown in Fig.~\ref{fig:SLy}. The median and maximum
    posterior estimations change, but the width of 90\% credible region remains
    insensitive to the choice of EOS. 

    If multiple EOSs are assumed, an intrinsic scatter will be introduced in the injected $\Lambda$ and $Q$. The effect of such scatter to the inference results is of the same order as the uncertainties shown in Fig.~\ref{fig:corner} and is much smaller to the constraints under single EOS assumption. 
    \ZW{I think it would be better to show the posterior distributions for APR4
    and SLy in the same figure. We expect the two posteriors to be largely overlapping, which would support our claim that the choice of EOS does not significantly affect the constraints.}
\end{authorresponse}

\begin{figure}[H]
    \centering
    \includegraphics[width=\textwidth]{hierarchical\_results\_SLy\_2d.pdf}
    \caption{Constraints to Love-Q relation in a linear model for APR4 and SLy.}
    \label{fig:SLy}
\end{figure}

\begin{reviewercomment}{6}
    The authors claim that the linear parameterization is sufficient for the inferred relations. How much of this is physical, and how much of this is a parameterization artifact? Can the authors clarify why, for example, in Fig.~2, the parameters $a_2$ and $b_2$ are already heavily correlated, as in the other parameters in the other parameterizations? Do other parameterizations also exist that are also plausible for use in this case (i.e. beyond polynomial forms)?
\end{reviewercomment}

\begin{authorresponse}
    We thank the reviewer for this constructive comment. To our knowledge, the
    polynomial forms are widely used parameterizations in universal relation
    studies, and there is no specific physical explanation for the model parameters.
    As shown in Figs.~2,4 and 5, for polynomial models with $j>2$, the
    posteriors extend to the prior boundary, while in the linear model case, the
    posteriors are well constrained within the prior range. This is the main reason why we claim that the linear model is sufficient for the inferred relations.

    We agree that there exist correlations in the linear model parameters, which
    means there is still possible parameter redundancy in the model. To further
    investigate this correlation, we perform a principal component analysis on
    the posterior samples. By reparameterizing the model along the dominant
    direction, we construct a one-parameter model \ZW{Plz specify the reparameterization procedure and the new parameter definition} and
    reconduct the inference. The results are shown in
    Fig.~\ref{fig:1d_inference}. For the region with $\Lambda \sim 300$, the
    constraints is significantly tighter compared with the linear model and
    other polynomial models. However, this only comes from the simplified model
    with one parameter, raising overfitting concerns. Also, without the linear
    model as a reference, it is difficult to construct a one-parameter model
    with appropriate coefficients (the ``0.9848'' and ``-0.1738'' in the
    definition of $X$). Thus, we believe that the linear model with two
    parameters is a more reasonable choice, avoiding both overfitting and underfitting.

    Ref.~\cite{Pappas:2013naa} introduced a different parameterization form, but
    the authors pointed out that their results were equivalent to those using
    polynomial forms. In this case, adopting this model is equivalent to a
    parameter transformation of current parameterization, and has little impact
    on the fitting itself, especially for the model dimension.

    \begin{revisedtext}
        ``To investigate the correlation between $a_2$ and $b_2$, we performed a Principal Component Analysis on the posterior samples of the linear model. By reparameterizing the model along the dominant principal axis, we conducted another hierarchical inference with a single hyperparameter. The corresponding results are presented in Figure 4. Notably, the right panel shows that the constraints almost vanish at $\Lambda \sim 300$. This single-parameter model can be regarded as the limiting case that $a_2$ and $b_2$ are completely correlated. A comparison with the results of linear model reveals that this correlation is linked to the ``hourglass-shaped'' structure of Love-Q relation constraints.''
    \end{revisedtext}
\end{authorresponse}
\ZW{I'm not sure to add the 1-d model in the manuscript (and the revision), since it may be too
technical and not very informative. But it is a good supplement to the
reviewer's comment, and we can add it only in the reply to the reviewer. For
other parameterizations, we can just briefly mention them in the manuscript, and
claim our methodology is applicable to them as well (if so).}

\begin{figure}
    \begin{minipage}[t]{0.46\textwidth}
    \centering
    \includegraphics[width=0.8\linewidth]{Hyper_parameter_1d_only.pdf}
    \end{minipage}
    \hfill
    \begin{minipage}[t]{0.52\textwidth}
    \includegraphics[width=\linewidth]{hierarchical_results_APR_1d.pdf}
    \end{minipage}
        \caption{\label{fig:1d_inference} Left panel: posterior distribution of the reparameterized model parameter $X=0.9848(a-\bar{a}) - 0.1738(b-\bar{b})$. Right panel: Recovered Love-Q relation from the posterior samples of $X$.} 
    \end{figure}

\begin{reviewercomment}{7}
    How relevant is the choice of a singular waveform model here in the results presented? Can waveform systematics play a role in the posteriors recovered?
\end{reviewercomment}

\begin{authorresponse}
    The waveform contains amplitude tidal corrections and higher-order spin-squared and spin-cubed terms at 3.5PN along with their corresponding spin-induced quadrupole moments, in addition to the spin-induced quadrupole moment terms at 2PN and 3PN~\cite{Samajdar:2020xrd}. These corrections or terms are important for our results since we're considering the inference of $\Lambda$ and $Q$. IMRPhenomXAS\_NRTidalv3 is an approximate example with a relatively high accuracy among current models.

    We completely agree that for GW events with SNR $\sim 10^3$, waveform systematics will likely become an important source of error, leading to bias in the posteriors of parameters like $\Lambda$. Following Ref.~\cite{Cutler:2007mi}, we estimate the error introduced by waveform inaccuracy
    \begin{equation}
        \Delta \theta^{i} = (\Gamma^{-1}(\theta_{\mathrm{bf}}))^{ij} \langle \partial_j h(\theta_{\mathrm{bf}}) | h_0(\theta_{\mathrm{tr}}) - h(\theta_{\mathrm{tr}})\rangle
    \end{equation}
    where $\Gamma$ is the fisher matrix, $\theta_{\mathrm{bf}}$ is the best-fit
    parameter, $\theta_{\mathrm{tr}}$ is the true parameter, and $h$ and $h_0$
    are GW strains assuming two different waveforms. Our calculation with the
    parameter combinations of the single events indicate that these errors
    introduced by waveform inaccuracy can be of the same order of magnitude as
    the statistical uncertainties for the single event inference. Specifically,
    the error is $\sim 30$ for $\Lambda$ and $\sim 0.5$ for $Q$.\ZW{Firstly, plz
    claim the models that you compare. Secondly, since the systematic error is
    comparable to the statistical uncertainty, it would be better to admit this
    limitation while still emphasizing the value of our work.} 
    In this work, we adopt a single waveform model for both the injections and
    the inference, thus the waveform systematics do not
    affect the main conclusion, which focuses on a methodology
    reference and an order-of-magnitude estimation. However, when dealing with
    real data with high SNRs in the future, it is important to consider the impact of waveform systematics.

    Considering the reviewer's comment, we have added a comment in the
    discussion as
    \begin{revisedtext}
        ``The inaccuracy of waveform itself can also introduce biases in the posteriors of single event parameters, especially for events with high SNRs [102].''
    \end{revisedtext}
\end{authorresponse}


\section*{Minor commments and questions}
\begin{reviewercomment}{1}
    On page 2, ``Future next-generation (XG) ground-based GW detectors...events per year''. Can this be clarified as to what would be the specific rate for BNSs?
\end{reviewercomment}

\begin{authorresponse}
    We thank the reviewer for pointing this out. The specific detection rate for BNS events is expected to be up to $10^5$--$10^6$ per year~\cite{Samajdar:2021egv}. We have updated the text to provide this specific rate for BNS coalescence.

    \begin{originaltext}
        ``Future next-generation (XG) ground-based GW detectors,..., are
        expected to detect many more GW signals, up to about $10^5$--$10^6$ events per
        year [60-63], thanks to their increased sensitivity and lower cutoff
        frequencies.'' 
    \end{originaltext}

    \begin{revisedtext}
        ``Future next-generation (XG) ground-based GW detectors,..., are expected to detect many more GW signals, up to about $10^5$--$10^6$ events per year for BNS coalescence [60-63], thanks to their increased sensitivity and lower cutoff frequencies.''
    \end{revisedtext}
\end{authorresponse}


\begin{reviewercomment}{2}
    On page 3, ``Yagi and Yunes fit the Love-Q relation with a quadratic polynomial model...'' Isn't this supposed to be quintic instead of quadratic?
\end{reviewercomment}

\begin{authorresponse}
    Yes, this is a typo. It should be quartic. Thanks for pointing it out.
    \begin{revisedtext}
        ``Yagi and Yunes fit the Love-Q relation with a quartic polynomial model...''
    \end{revisedtext}
\end{authorresponse}


\begin{reviewercomment}{3}
    On page 4, ``Similarly, the EOS parameters determine the relation between $\Lambda$ and $m$, ...'' The phrasing of this entire statement is a bit awkward. 
\end{reviewercomment}

\begin{authorresponse}
    We agree that the original phrasing had a problem. We have rephrased this sentence.

    \begin{originaltext}
        ``Similarly, the EOS parameters determine the relation between $\Lambda$ and $m$,
        as well as between $Q$ and $m$, can be regarded as hyperparameters and analyzed
        in the hierarchical Bayesian framework.'' 
    \end{originaltext}
    
    \begin{revisedtext}
        ``Similarly, the EOS parameters determining $\Lambda$-$m$ and $Q$-$m$ relations
        can be regarded as hyperparameters and analyzed in the hierarchical Bayesian framework.''
    \end{revisedtext}
\end{authorresponse}


\begin{reviewercomment}{4}
    On page 4, ``Since we are inferring the Love-Q relation, the tidal deformabilities...'' Why treat the tidal deformabilities of a single event as independent when the recovered parameter is usually of the form of the effective tidal deformability $\tilde{\Lambda}$?
\end{reviewercomment}

\begin{authorresponse}
    The SNR is high, thus the selection of priors does not matter. Choosing $\Lambda_i$ and $Q_i$ as the independent parameters and adopting a flat prior for them might be a more direct choice. A flat prior of $\Lambda_i$ and $Q_i$ also allows us to simplify Eq.~(2.14) as Eq.~(2.15) in the manuscript.

    \begin{originaltext}
        ``Since we are inferring the Love-Q relation, the tidal deformabilities, $\bm
        {\Lambda}_i=\{\Lambda_{1i},\Lambda_{2i}\}$, and quadrupole moments,
        $\bm{Q}_i=\{Q_ {1i},Q_{2i}\}$, of the two NSs of the $i\text{-th}$ event are
        treated as independent parameters in $\bm{\theta}_i$.'' 
    \end{originaltext}

    \begin{revisedtext}
        ``We treat the tidal deformabilities, $\bm{\Lambda}_i=\{\Lambda_{1i},\Lambda_{2i}\}$, 
        and quadrupole moments, $\bm{Q}_i=\{Q_ {1i},Q_{2i}\}$, of the two NSs of the $i\text{-th}$ event 
        as independent parameters in $\bm{\theta}_i$ and select a flat prior for them. This is a direct choice 
        since the prior is not important for high-SNR event inference.''
    \end{revisedtext}

\end{authorresponse}


\begin{reviewercomment}{5}
    On page 6, maybe change ``merger'' time $t_c$ to ``coalescence'' time for consistency with the phase of coalescence $\phi_c$. 
\end{reviewercomment}

\begin{authorresponse}
    We agree that ``coalescence time'' is a better expression considering the context.
\end{authorresponse}


\begin{reviewercomment}{6}
    On page 6, There are several technical reports of the Einstein Telescope (ET) and Cosmic Explorer (CE) placed at several proposed sites. Why assume the current locations of LIGO and Virgo? 
\end{reviewercomment}

\begin{authorresponse}
    We acknowledge that the final sites for ET and CE are still under evaluation and will likely differ from the current LIGO and Virgo locations. However, for the purpose of this study, we assumed the current locations as a representative configuration. The exact location choice of the detectors has a marginal impact on the core results and conclusions of our study.
\end{authorresponse}


\begin{reviewercomment}{7}
    On page 7, what are the range of SNR values for the highest-SNR events selected in this study? 
\end{reviewercomment}

\begin{authorresponse}
    In our study, the highest SNR values range from about 1500 to 2500. The sources have luminosity distances near the lower bound 15 Mpc. The significantly lower noise level of next generation detectors lead to a high SNR for these nearby simulated sources.
\end{authorresponse}


\begin{reviewercomment}{8}
    On page 10, Fig.~4, the blue bands of the right panel have this curving behavior at large $\Lambda$ (and also at small $\Lambda$). Why is this the case? 
\end{reviewercomment}

\begin{authorresponse}
    The constraints in the middle where most data points gather are insensitive to the parameterization of Love-Q relation. Under this precondition, the polynomial models with more degrees of freedom naturally span a wider region in the two ends compared with linear case. 
\end{authorresponse}


\begin{reviewercomment}{9,10}
    On page 11, ``$\theta$ and $\beta$ is taken to be dimensionless...'' Replace ``is'' with ``are''. 
    
    On page 12, ``$\alpha$ has the dimension of length square...'' Replace ``square'' with ``squared''.
\end{reviewercomment}

\begin{authorresponse}
    We apologize for these mistakes. We thank the reviewer for correcting them.
\end{authorresponse}


\begin{thebibliography}{99}
    \bibitem{LIGOScientific:2016}
    B.~P.~Abbott \textit{et al.} [LIGO Scientific and Virgo],
    \emph{Binary Black Hole Mergers in the first Advanced LIGO Observing Run},
    \emph{Phys. Rev. X} {\textbf{6}}, no.4, 041015 (2016) [arXiv:1606.04856 [gr-qc]].

    
    \bibitem{Mandel2018}
    I. Mandel, W. M. Farr and J. R. Gair,
    \emph{Extracting distribution parameters from multiple uncertain observations with selection biases},
    \emph{Mon. Not. Roy. Astron. Soc.} {\bf 486} (2019) 1086 [arXiv:1809.02063 [physics.data-an]].
    
    \bibitem{Thrane2019}
    Thrane, Eric and Talbot, Colm,
    \emph{An introduction to Bayesian inference in gravitational-wave astronomy: Parameter estimation, model selection, and hierarchical models},
    \emph{Publications of the Astronomical Society of Australia, Cambridge University Press (CUP)} {\bf 36} (2019) 1086 [arXiv:1809.02063].

    \bibitem{Pani:2015nua}
    P. Pani, L. Gualtieri and V. Ferrari,
    \emph{Tidal Love numbers of a slowly spinning neutron star},
    \emph{Phys. Rev. D} {\textbf{92}}, no.12, 124003 (2015) [arXiv:1509.02171 [gr-qc]].

    \bibitem{Yagi:2014bxa}
    K.~Yagi, K.~Kyutoku, G.~Pappas, N.~Yunes and T.~A.~Apostolatos,
    \emph{Effective No-Hair Relations for Neutron Stars and Quark Stars: Relativistic Results},
    \emph{Phys. Rev. D} \textbf{89}, no.12, 124013 (2014) [arXiv:1403.6243 [gr-qc]].

    \bibitem{Doneva:2013rha}
    D.~D.~Doneva, S.~S.~Yazadjiev, N.~Stergioulas and K.~D.~Kokkotas,
    \emph{Breakdown of I-Love-Q universality in rapidly rotating relativistic stars},
    \emph{Astrophys. J. Lett.} {\textbf{781}}, L6 (2013) [arXiv:1310.7436 [gr-qc]].

    \bibitem{Pappas:2013naa}
    G.~Pappas and T.~A.~Apostolatos,
    \emph{Effectively universal behavior of rotating neutron stars in general relativity makes them even simpler than their Newtonian counterparts},
    \emph{Phys. Rev. Lett.} \textbf{112}, 121101 (2014) [arXiv:1311.5508 [gr-qc]].

    \bibitem{Doneva:2014faa}
    D.~D.~Doneva, S.~S.~Yazadjiev, K.~V.~Staykov and K.~D.~Kokkotas,
    \emph{Universal I-Q relations for rapidly rotating neutron and strange stars in scalar-tensor theories},
    \emph{Phys. Rev. D} \textbf{90}, no.10, 104021 (2014) [arXiv:1408.1641 [gr-qc]].

    \bibitem{Chakrabarti:2013tca}
    S.~Chakrabarti, T.~Delsate, N.~G{\"u}rlebeck and J.~Steinhoff,
    \emph{I-Q relation for rapidly rotating neutron stars},
    \emph{Phys. Rev. Lett.} \textbf{112}, 201102 (2014) [arXiv:1311.6509 [gr-qc]].

    \bibitem{Samajdar:2020xrd}
    A.~Samajdar and T.~Dietrich,
    \emph{Constructing Love-Q-Relations with Gravitational Wave Detections},
    \emph{Phys. Rev. D} \textbf{101}, no.12, 124014 (2020) [arXiv:2002.07918 [gr-qc]].

    \bibitem{Cutler:2007mi}
    C.~Cutler and M.~Vallisneri,
    \emph{LISA detections of massive black hole inspirals: Parameter extraction errors due to inaccurate template waveforms},
    \emph{Phys. Rev. D} \textbf{76}, 104018 (2007) [arXiv:0707.2982 [gr-qc]].

    \bibitem{Samajdar:2021egv}
    A.~Samajdar, J.~Janquart, C.~Van Den Broeck and T.~Dietrich,
    \emph{Biases in parameter estimation from overlapping gravitational-wave signals in the third-generation detector era},
    \emph{Phys. Rev. D} \textbf{104}, no.4, 044003 (2021) [arXiv:2102.07544 [gr-qc]].
    

    
\end{thebibliography}
\end{document}